\documentclass[12pt,a4paper]{article}
\usepackage{amsmath,amssymb}
\usepackage[margin=0.5in]{geometry} % custom margins
\usepackage{graphicx}
\usepackage{hyperref}
\usepackage{minted}
\usepackage{caption} % custom captions
\usepackage{float} % place figures as float
\usepackage{circuitikz} % circuits, also loads tikz

% When writing indented paragraphs:
% \usepackage{indentfirst}

% To supress page numbers:
% \usepackage{nopageno}

\begin{document}
\begin{center}
	\includegraphics[width=\textwidth]{./Images/Header.jpeg}
	\vfill		
	\textbf{\Large{Report for Experiment \#N\\
	Lab Name}}
	\vfill
	Name\\
	Lab Partner: Name\\
	TA: Name\\
	Date
	\vfill
\end{center}
	
\section*{Abstract:}
	Summarize motivation and main results.
	
\newpage
	
\section*{Introduction:}
	Sample equations,
	\[ \vec{\nabla} \cdot \vec{E} = \frac{\rho}{\varepsilon_0} \tag{1} \]
	\[ \vec{\nabla} \cdot \vec{B} = 0 \tag{2} \]
	\[ \vec{\nabla} \times \vec{E} = - \frac{\partial \vec{B}}{\partial t} \tag{3} \]
	\[ \vec{\nabla} \times \vec{B} = \mu_0 \left( \vec{J} + \varepsilon_0 \frac{\partial \vec{E}}{\partial t} \right) \tag{4} \]
	Within report reference as [\textit{Equation n}].
\section*{Investigation 1:}	
	Sample derived equation,
	\[ z = f(x_1,x_2,...,x_n) 
	\hspace{1cm} \Rightarrow \hspace{1cm}
	\delta z = \sqrt{\sum_{i=1}^n \left(
	\frac{\partial f(x_1,x_2,...,x_n)}{\partial x_i} \delta x_i \right)^2 }\]
	Sample table,
	\begin{figure}[H]
		\centering
		\begin{tabular}{ccccc}
			\(\displaystyle \vec{F}_E = \frac{1}{4 \pi \varepsilon_0} \frac{qQ}{r^2} \hat{r} \) &
			$\rightarrow$ & 
			$\displaystyle \vec{E} = \frac{1}{q} \vec{F}$ & 
			$\rightarrow$ &
			\(\displaystyle \vec{E} = \frac{1}{4 \pi \varepsilon_0} \frac{Q}{r^2} \hat{r} \)\\
			$\uparrow$ & & & & $\downarrow$\\
			\(\displaystyle \vec{F}_E = - \vec{\nabla} \Delta U\) & & & &
			\(\displaystyle \Delta V = - \int_C \vec{E} \cdot d\vec{l}\)\\
			$\uparrow$ & & & & $\downarrow$\\
			\(\displaystyle \Delta U = \frac{1}{4 \pi \varepsilon_0} \frac{Qq}{r} \) &
			$\leftarrow$ & 
			$\Delta U = q \Delta V$ & 
			$\leftarrow$ &
			\(\displaystyle \Delta V = \frac{1}{4 \pi \varepsilon_0} \frac{Q}{r} \)
		\end{tabular}
		\caption{Random Table}
	\end{figure} \noindent
	Sample code,
	Sample circuit with Circuitikz,
	\begin{figure}[H]
		\centering
		\begin{circuitikz}
	    	\draw (0,0)
	      	to[V,v=$V(t)$] (0,2) % The voltage source
		to[short] (2,2)
	      	to[R=$R_1$] (2,0) % The resistor
	      	to[short] (0,0);
	      	\draw (2,2)
	      	to[short] (4,2)
	      	to[L=$L_1$] (4,0)
	      	to[short] (2,0);
	      	\draw (4,2)
	      	to[short] (6,2)
	      	to[C=$C_1$] (6,0)
	      	to[short] (4,0);
		\end{circuitikz}
  		\caption{Random Circuit}
	\end{figure}				
		
\section*{Conclusion:}
	I'm never really sure what goes here.

\section*{Questions:}
	\begin{enumerate}
		\item Question 1
	\end{enumerate}
	
\section*{References:}
	\begin{enumerate}
		\item \href{https://www.tablesgenerator.com/#}{Table Generator \LaTeX}, helpful if converting Excel Data
		\item \href{https://web.northeastern.edu/ipl/data-analysis/straight-line-fit/}{Northeastern IPL Straight Line Fit Calculator}
	\end{enumerate}
\end{document}
